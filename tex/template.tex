\documentclass[a4paper, 12pt]{exam}

\usepackage{graphicx}
\usepackage{caption}
\usepackage[left=1.5cm, top=2.5cm, right=1.5cm, bottom=1.5cm, nohead, includefoot]{geometry}
\usepackage{float}
\usepackage{hyperref}


\renewcommand{\baselinestretch}{1.5}

\newcommand{\myRunningheader}{
\hrule\vspace{0.5em}
تمرین \تمرین \پررا مهلت تحویل: \مهلت \پررا مدرس: دکتر حسین اسدی
\vspace{0.5em}\hrule
}

\newcommand{\myFirstheader}{
\begin{minipage}[t]{.98\textwidth}
\begin{minipage}[t]{.33\textwidth}
سیستم عامل\\
فلان\\
بیسار\\
تمرین بیسار
\end{minipage}
\hfill
\begin{minipage}[t]{.33\textwidth}
\vspace{-5ex}
\centering
\includegraphics[width=.5\textwidth]{logo.pdf}\\
دانشگاه صنعتی شریف\\
دانشکده مهندسی کامپیوتر
\end{minipage}
\end{minipage}
}

\pagestyle{headandfoot}
\lhead{}
\chead[]{\myRunningheader}
\rhead{}
\lfoot{}
\cfoot{صفحه\thepage~از~\numpages}
\rfoot{}

\hypersetup{
	colorlinks=true,
	breaklinks,
	urlcolor=blue,
	linkcolor=blue
}

%%%%%%%%%%%%%%%%%%%%%%%%%%%%%%%%%%%%%%%%%%%%%%%%%%%%%%
\usepackage[localise]{xepersian}
%\settextfont{HM XNiloofar}
\settextfont{HM FNazli}
%\setlatintextfont{...}

\pointpoints{نمره}{نمره‌}
\فرمان‌جانشین{چر}{متن‌لاتین}
\فرمان‌جانشین{رچ}{متن‌فارسی}
\فرمان‌جانشین{سایت}{href}

\فرمان‌نو{\سایت‌درس}{http://cw.sharif.ir/course/view.php?id=2549}

\فرمان‌نو{\مهلت}{۲۲ مهر ۱۳۹۰}
\فرمان‌نو{\نام‌تمرین}{\چر{Assignment 1}}
\فرمان‌نو{\تمرین}{اول}



\begin{document}
\myFirstheader

\centering
\fbox{
	\begin{minipage}{0.92\textwidth}
		\vspace{.5em}
		\متن‌سیاه{هدف از تمرین:} آشنایی با نحوه زمان‌بندی پردازه‌ها
		
		\متن‌سیاه{مهلت تحویل:} \مهلت
		
		\متن‌سیاه{نحوه‌ی تحویل:} پاسخ‌های تمرین خود را حداکثر تا ساعت ۲۳:۵۹ روز \مهلت~ به صورت فایل فشرده در فرمت \چر{ZIP} و یا \چر{RAR}
		بر روی \سایت{\سایت‌درس}{صفحه‌ی درس} در بخش مربوط به \نام‌تمرین~ بارگذاری کنید.
		فایل‌فشرده شما باید شامل پاسخ سوالات تئوری به صورت تایپ شده و یا اسکن شده،
		کد‌های مربوط به سوالات‌برنامه‌نویسی و \متن‌سیاه{گزارشی} شامل توضیح کد (در سوالات برنامه نویسی) و نحوه‌ی اجرای آن باشد.
		توجه‌کنید که تمامی کدها و فایل‌های مورد نیاز اجرای آن برای هر سوال در پوشه‌ای مجزا با نام‌گذاری به صورت $Q\#$ (به طور مثال $Q5$) قرار گیرد.
		
		\متن‌سیاه{نکات‌ضروری:} تمامی بخش‌های تمرین و کدهای شما باید توسط خود شما نوشته شده باشد،
		در غیر این صورت نظر مطابق قوانین کلاس برخورد خواهد شد.
		در صورت دیرکرد تا حداکثر دو روز بعد از \مهلت، می‌توانید فایل تمرین خورد را بر روی صفحه‌ی درس بارگذاری کنید.
		
		\تنظیم‌ازوسط{در صورتی که هر پرسشی در رابطه با تمرین دارید، \متن‌سیاه{\درشت{تنها}} از طریق \سایت{\سایت‌درس}{فروم درس} و در بخش مربوطه مطرح کنید.}
		\vspace{.5em}
	\end{minipage}
}

\بخش*{سوالات تئوری}
\begin{questions}
	
	\question[10]
	این یک سوال نمونه با تعیین نمره است.
	\begin{parts}
		\part
		این یک زیرسوال است.
		\begin{subparts}
			\subpart
			این یک زیرزیرسوال است.
		\end{subparts}
		
	\end{parts}
	\question
	این یک سوال بدون تعیین نمره آن است.
	لورم ایپسوم یا طرح‌نما (به انگلیسی: Lorem ipsum) به متنی آزمایشی و بی‌معنی در صنعت چاپ، صفحه‌آرایی و طراحی گرافیک گفته می‌شود. طراح گرافیک از این متن به عنوان عنصری از ترکیب بندی برای پر کردن صفحه و ارایه اولیه شکل ظاهری و کلی طرح سفارش گرفته شده استفاده می نماید، تا از نظر گرافیکی نشانگر چگونگی نوع و اندازه فونت و ظاهر متن باشد. معمولا طراحان گرافیک برای صفحه‌آرایی، نخست از متن‌های آزمایشی و بی‌معنی استفاده می‌کنند تا صرفا به مشتری یا صاحب کار خود نشان دهند که صفحه طراحی یا صفحه بندی شده بعد از اینکه متن در آن قرار گیرد چگونه به نظر می‌رسد و قلم‌ها و اندازه‌بندی‌ها چگونه در نظر گرفته شده‌است. از آنجایی که طراحان عموما نویسنده متن نیستند و وظیفه رعایت حق تکثیر متون را ندارند و در همان حال کار آنها به نوعی وابسته به متن می‌باشد آنها با استفاده از محتویات ساختگی، صفحه گرافیکی خود را صفحه‌آرایی می‌کنند تا مرحله طراحی و صفحه‌بندی را به پایان برند.
	
	\question
	لورم ایپسوم یا طرح‌نما (به انگلیسی: Lorem ipsum) به متنی آزمایشی و بی‌معنی در صنعت چاپ، صفحه‌آرایی و طراحی گرافیک گفته می‌شود. طراح گرافیک از این متن به عنوان عنصری از ترکیب بندی برای پر کردن صفحه و ارایه اولیه شکل ظاهری و کلی طرح سفارش گرفته شده استفاده می نماید، تا از نظر گرافیکی نشانگر چگونگی نوع و اندازه فونت و ظاهر متن باشد. معمولا طراحان گرافیک برای صفحه‌آرایی، نخست از متن‌های آزمایشی و بی‌معنی استفاده می‌کنند تا صرفا به مشتری یا صاحب کار خود نشان دهند که صفحه طراحی یا صفحه بندی شده بعد از اینکه متن در آن قرار گیرد چگونه به نظر می‌رسد و قلم‌ها و اندازه‌بندی‌ها چگونه در نظر گرفته شده‌است. از آنجایی که طراحان عموما نویسنده متن نیستند و وظیفه رعایت حق تکثیر متون را ندارند و در همان حال کار آنها به نوعی وابسته به متن می‌باشد آنها با استفاده از محتویات ساختگی، صفحه گرافیکی خود را صفحه‌آرایی می‌کنند تا مرحله طراحی و صفحه‌بندی را به پایان برند.
	
	\question
	لورم ایپسوم یا طرح‌نما (به انگلیسی: Lorem ipsum) به متنی آزمایشی و بی‌معنی در صنعت چاپ، صفحه‌آرایی و طراحی گرافیک گفته می‌شود. طراح گرافیک از این متن به عنوان عنصری از ترکیب بندی برای پر کردن صفحه و ارایه اولیه شکل ظاهری و کلی طرح سفارش گرفته شده استفاده می نماید، تا از نظر گرافیکی نشانگر چگونگی نوع و اندازه فونت و ظاهر متن باشد. معمولا طراحان گرافیک برای صفحه‌آرایی، نخست از متن‌های آزمایشی و بی‌معنی استفاده می‌کنند تا صرفا به مشتری یا صاحب کار خود نشان دهند که صفحه طراحی یا صفحه بندی شده بعد از اینکه متن در آن قرار گیرد چگونه به نظر می‌رسد و قلم‌ها و اندازه‌بندی‌ها چگونه در نظر گرفته شده‌است. از آنجایی که طراحان عموما نویسنده متن نیستند و وظیفه رعایت حق تکثیر متون را ندارند و در همان حال کار آنها به نوعی وابسته به متن می‌باشد آنها با استفاده از محتویات ساختگی، صفحه گرافیکی خود را صفحه‌آرایی می‌کنند تا مرحله طراحی و صفحه‌بندی را به پایان برند.
\end{questions}

\بخش*{سوالات عملی}
\begin{questions}
	\question
	لورم ایپسوم یا طرح‌نما (به انگلیسی: Lorem ipsum) به متنی آزمایشی و بی‌معنی در صنعت چاپ، صفحه‌آرایی و طراحی گرافیک گفته می‌شود. طراح گرافیک از این متن به عنوان عنصری از ترکیب بندی برای پر کردن صفحه و ارایه اولیه شکل ظاهری و کلی طرح سفارش گرفته شده استفاده می نماید، تا از نظر گرافیکی نشانگر چگونگی نوع و اندازه فونت و ظاهر متن باشد. معمولا طراحان گرافیک برای صفحه‌آرایی، نخست از متن‌های آزمایشی و بی‌معنی استفاده می‌کنند تا صرفا به مشتری یا صاحب کار خود نشان دهند که صفحه طراحی یا صفحه بندی شده بعد از اینکه متن در آن قرار گیرد چگونه به نظر می‌رسد و قلم‌ها و اندازه‌بندی‌ها چگونه در نظر گرفته شده‌است. از آنجایی که طراحان عموما نویسنده متن نیستند و وظیفه رعایت حق تکثیر متون را ندارند و در همان حال کار آنها به نوعی وابسته به متن می‌باشد آنها با استفاده از محتویات ساختگی، صفحه گرافیکی خود را صفحه‌آرایی می‌کنند تا مرحله طراحی و صفحه‌بندی را به پایان برند.
	
	
	\question
	لورم ایپسوم یا طرح‌نما (به انگلیسی: Lorem ipsum) به متنی آزمایشی و بی‌معنی در صنعت چاپ، صفحه‌آرایی و طراحی گرافیک گفته می‌شود. طراح گرافیک از این متن به عنوان عنصری از ترکیب بندی برای پر کردن صفحه و ارایه اولیه شکل ظاهری و کلی طرح سفارش گرفته شده استفاده می نماید، تا از نظر گرافیکی نشانگر چگونگی نوع و اندازه فونت و ظاهر متن باشد. معمولا طراحان گرافیک برای صفحه‌آرایی، نخست از متن‌های آزمایشی و بی‌معنی استفاده می‌کنند تا صرفا به مشتری یا صاحب کار خود نشان دهند که صفحه طراحی یا صفحه بندی شده بعد از اینکه متن در آن قرار گیرد چگونه به نظر می‌رسد و قلم‌ها و اندازه‌بندی‌ها چگونه در نظر گرفته شده‌است. از آنجایی که طراحان عموما نویسنده متن نیستند و وظیفه رعایت حق تکثیر متون را ندارند و در همان حال کار آنها به نوعی وابسته به متن می‌باشد آنها با استفاده از محتویات ساختگی، صفحه گرافیکی خود را صفحه‌آرایی می‌کنند تا مرحله طراحی و صفحه‌بندی را به پایان برند.
\end{questions}

\end{document}